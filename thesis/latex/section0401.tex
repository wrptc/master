\chapter{Experiment}
\section{Preprocessing}
We use four different kinds of data in our experiment:\\
Experiment 1: prove the feasibility of KeyGraph Algorithm.\\
Experiment 2: Using the whole set of document to experiment and assess the runtime of proposed algorithm.\\
Experiment 3: a set of two short Japanese stories, and set $\tau = 1$.\\
Experiment 4-1: a set of four similar precedent documents, and set $\tau = 1$.\\
Experiment 4-2: a set of four different precedent documents, and set $\tau = 1/2$ .\\
Experiment 4-3: a set of three precedent documents include similar and different ones, and set $\tau = 2/3$.\\
First, we need to transform the code of text to utf-8, we use Python language:\\
\begin{enumerate}[*]
\item import codecs
\item infile = codecs.open(`input',`r',`Shift JIS')
\item outfile = codecs.open(`output',`w',`utf-8')
\end{enumerate}
Then the files can be accessed by KNP system, we also need to piece the document to sentences to get a correct result from KNP system:\\
\begin{enumerate}[*]
\item sentences = document.split(`\begin{CJK}{UTF8}{ipxm}。\end{CJK}')
\end{enumerate}
Here, we use `\begin{CJK}{UTF8}{ipxm}。\end{CJK}' as a symbol of the end of a sentence. Because of precedents contains only declarative sentence, this simple way can solve the problem very fast and easily.
Then we can use KNP system to analysis our sentences with the command:\\
\begin{enumerate}[*]
\item `sentence' $|$ juman $|$ knp -simple
\end{enumerate}

For each document, we have a file to save KNP result.\\
Here are examples from the KNP result:\\
\begin{enumerate}[*]
\item \begin{CJK}{UTF8}{ipxm}
`目録 もくろく 目録 名詞 6 普通名詞 1 * 0 * 0 "代表表記:目録/もくろく カテゴリ:抽象物" '
\end{CJK}
\end{enumerate}
from lines of this kind, we can extract all nouns and use them to extract keyword using KeyGraph Algorithm.\\
\begin{enumerate}[*]
\item \begin{CJK}{UTF8}{ipxm}
`<格解析結果:操業/そうぎょう:動7:ガ/N/被告/0/0/1;...'
\end{CJK}
\end{enumerate}
from lines of this kind, we can extract the events.\\
With the keywords and all the events, we can start our experiment.\\