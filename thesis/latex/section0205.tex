\section{Beam Search}
Beam Search is a heuristic graph search algorithm, usually used in the case of very large solution space, in order to reduce the space occupied by the search space and time, in each step of the depth of the expansion, cut off some of the poor quality of the node, keep some of the higher quality of the node. This reduces the space consumption and improves the efficiency of time, but the disadvantage is that there may be potential best solutions are discarded, so Beam Search algorithm is not complete.\\
Beam Search uses a breadth-first strategy to create a search tree that sorts the nodes at the heuristic cost at each level of the tree, and then leaves only nodes of a predetermined number (Beam Width) The node continues to expand at the next level, and the other nodes are cut off. If the beam width is infinite, then the search is the width of the first search. But the disadvantage of the pruning is that there may be some results that potential to be the best solution are discarded in the process above.Thus, the beam search algorithm is incomplete and uninfected. It is usually used in some large systems, such as machine translation systems, voice recognition systems, which can be very large and the only correct solution does not exist. The goal of this system is to use the fastest way to find the most appropriate solution.\\
As we have introduced before, beam search algorithm is a simplified method of the breadth-first algorithm. It has a heuristic function $h$ and a pre-set beam width $B$.\\
The heuristic function h is used to estimate the consumption from the given node to the target node, and the beam width $B$ is responsible for limiting the number of nodes that should be stored in each level of the breadth-first search. That is, the heuristic function allows the algorithm to select the node that directs it to the target node, and the beam width only allows the algorithm to store the important nodes in the memory and prevent it from being exhausted from the memory before finding the target node.\\
The processing is as follow.\\
\begin{enumerate}
\item Insert the initial node into the list,
\item The node will be heap if the node is the target node, the algorithm ends;
\item Otherwise expand the node, take the beam width of the node into the heap. Then go to the second step to continue the cycle.
\item The end of the algorithm is to find the optimal solution or the heap is empty.
\end{enumerate}
The beam width can be pre-set or changed, and you can search by a minimum beam width. If you do not find the appropriate solution, then expand the beam width and find it again.\\
Personally think that the beam search method, in fact, provides a way to find the best solution, that is, in the appropriate circumstances, you can cut some of the low credibility of the path, in actual use, you can each layer of the beam width is inconsistent, For example, in some of the initial level to retain some of the results, in the back can be assured that bold pruning. Of course, you can also live and use, you can combine the depth of priority algorithm, through backtracking, you can find the optimal solution.