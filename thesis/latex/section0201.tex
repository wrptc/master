\chapter{Prerequisite Knowledge}
\section{Precedent}
Japanese new constitution started from 1947. The re-establishment of the whole legal system also began in 1947 to conform to the new constitutional principles. The Constitution in the three basic characteristics features. First, it transplants from US customary law system. Second, it overthrew the emperor of the divine monarchy, established a guarantee for modern civil liberties, in accordance with the principles of democracy legal system. Third, article ninth of the constitution to renounce war and military, to become the ``peace constitution''.\\
Under the Japanese civil law system, court cases provide the criteria for how the law should be interpreted in reality. Although without judicial restraint, the judge also seriously considers the case, especially the Supreme Court's decision, which makes the understanding of precedent become the basis of the implementation of the law.\\
In our research, we use a precedent database provided by D1-Law[3], DAI-ICHI HOKI co., Ltd. This database is contracted with Hokkaido University and can be accessed from within the university network. In addition, ``Court case information'' on the court website, provided by Westlaw Japan Co. There is an available database such as ``Japanese law general online service'', Supreme Court judicial precedent explanation and others. In each case, it is possible to search using case number, incident name, trial judgment date, etc. Considering the number of judicial cases, ease of access, search function, etc. In our research, we decided to use ``D1law.com.'' There are some features of the database.
\begin{enumerate}
\item Japanese largest precedent database with over 210,000 precedents and over 29000 legal provisions.
\item Advantage search supported by the database, we can easily search precedents from the point of legal provisions.
\end{enumerate}
In D1-Law database, we can find some important information of each precedent: 
\begin{enumerate}
\item Basic Information: Precedent ID, Date, Results, etc. 
\item Precedent Main Text: the explanation of the case includes Main Text, and Facts and Reasons.
\item Precedent Abstract: a summarization by the specialist from legal provision view. 
\end{enumerate}
The Precedent Main Text part is the main part of a case. The part Main Text is a part of a brief description of the case. The part Facts and Reasons contains long text and detailed description.\\
The part Facts and Reasons is not a specific configuration required in a case. According to an unwritten customary, almost all precedents have this part as a fact. This part contains four kinds of information: Claim, Indisputable fact, Dispute and Propositions, and Court decision.\\
In our research, we mainly use Precedent Main Text as documents to study descriptive similarity among them.\\
Here is a part of list of the precedents we used in our research(20 of 100).\\
\newpage
\begin{table}[!h]
\centering
\begin{tabular}{ccc}
\hline
Serial number&Precedent ID&Precedent Name\\
001&25000036&\begin{CJK}{UTF8}{ipxm}損害賠償請求上告事件\end{CJK}\\
002&25000038&\begin{CJK}{UTF8}{ipxm}木曽駒高原眺望権訴訟\end{CJK}\\
003&35000044&\begin{CJK}{UTF8}{ipxm}釧路違法公正証書損害賠償請求訴訟\end{CJK}\\
004&27805492&\begin{CJK}{UTF8}{ipxm}国家賠償請求事件\end{CJK}\\
005&27807602&\begin{CJK}{UTF8}{ipxm}中和歌山観音竹商法損害賠償請求訴訟\end{CJK}\\
006&27807603&\begin{CJK}{UTF8}{ipxm}損害賠償請求事件\end{CJK}\\
007&27807929&\begin{CJK}{UTF8}{ipxm}沼津セクシュアル・ハラスメント訴訟\end{CJK}\\
008&27809649&\begin{CJK}{UTF8}{ipxm}損害賠償請求事件\end{CJK}\\
009&27811304&\begin{CJK}{UTF8}{ipxm}損害賠償請求控訴事件\end{CJK}\\
010&27811611&\begin{CJK}{UTF8}{ipxm}損害賠償請求事件\end{CJK}\\
011&27813101&\begin{CJK}{UTF8}{ipxm}日鉄鉱業松尾採石所じん肺訴訟控訴\end{CJK}\\
012&27814650&\begin{CJK}{UTF8}{ipxm}損害賠償請求控訴事件\end{CJK}\\
013&27816661&\begin{CJK}{UTF8}{ipxm}東京電力(群馬)事件\end{CJK}\\
014&27816925&\begin{CJK}{UTF8}{ipxm}損害賠償請求事件\end{CJK}\\
015&27818787&\begin{CJK}{UTF8}{ipxm}損害賠償請求事件\end{CJK}\\
016&27818788&\begin{CJK}{UTF8}{ipxm}勧角証券違法勧誘損害賠償請求事件\end{CJK}\\
017&27818866&\begin{CJK}{UTF8}{ipxm}損害賠償請求事件\end{CJK}\\
018&27820817&\begin{CJK}{UTF8}{ipxm}東京電力事件(山梨東電訴訟)\end{CJK}\\
019&27824746&\begin{CJK}{UTF8}{ipxm}損害賠償請求事件\end{CJK}\\
020&27825841&\begin{CJK}{UTF8}{ipxm}債務不存在確認請求-同反訴請求事件\end{CJK}\\
\hline
\end{tabular}
\end{table}